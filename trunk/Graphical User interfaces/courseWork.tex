\documentclass[a4paper,12pt]{article} %indique la classe du document, et les options

\pagenumbering{roman}
%le pr\'eambule
\usepackage[english]{babel}
\usepackage{graphicx}
\usepackage{array}
\usepackage{multicol}
\selectlanguage{english}


%Titre
\makeatletter
\def\maketitle{

	\begin{multicols}{2}
		{\@author\\\texttt{\@email}}
		\begin{flushright}
			{\includegraphics[width=0.5\linewidth]{../../Newcastle-University.jpg}}\\
			{\@date}\\
		\end{flushright}
	\end{multicols}
	\vspace{1cm}
	\begin{center}
		{\LARGE \@title}
		\rule{10cm}{1pt}
	\end{center}
	\vspace{1cm}
}
\def\email#1{\def\@email{#1}}
\makeatother
\email{nicolas.desfeux@gmail.com}
\date{\today}
\author{Nicolas Desfeux}
\title{
\Huge{Designing a Graphical User Interface for the Bracelet Computer
}}


%document principal 

\begin{document}
\maketitle

\pagenumbering{arabic}

\section*{Introduction}
\paragraph{}The purpose of this document is to explain the design choices and the working way of the bracelet computer. It will show the benefits and the drawbacks of choices, compare to other solutions. 
\paragraph{}I imagine the computer like as a tube, with items on it. To switch from one item to another, you just have to use the left and right arrows. Up and down arrows helps you navigates in the item. Push button is use for selection and validation. We will see the operations and the functionalities, and discuss the choises about them.

\section{Operation}
\subsection{Available actions}
\begin{itemize}
\item \textbf{Left and Right Arrows} :  Those actions allows to navigate between the different items of the computer (define a new place, going home, going to a particular place,...). 
\item \textbf{Up and Down arrows} : Allows navigation inside an item (switching current location display for example).
\item \textbf{Short push}
\item \textbf{Long push} A long is equivalent to holding the button for 3 seconds or more.
\end{itemize}
The idea of having a long push allows one more action. The time might be discussed, and can maybe be part of settings, and be changeable. Anyway, the idea is to keep a permanent and easy access to the way home. Anywhere you are in the computer, a long push will bring back to the screen that give you the way home.\\

\section{About functionalities}
\subsection{Current Location}
This screen tells you were you are. Here is two displays : one (default) show you the street you are, and the other other one show you the nearest known place you are (Home or relatives house for example). 
\subsection{Going Home}
We focused on keeping texts instead of graphics, as text is more explicit. Based on the Nielsen's for graphics.
\subsection{Going to a predefined place}
Select a place then go.
\subsection{Registered a new place}
List of predefine places. 
Password for erase (could be different).
\subsection{Maximum Distance}
password again.
\subsection{Too far}
vibrate
Allow to go home or nearest place.
\section*{Conclusion}
This design is provided for elderly people, with dementia. The operations of the computer have been choose to be very simple, and based on Nielsen's principles (especially on graphics parts). 

\end{document}